\documentclass{article}

\usepackage[utf8]{inputenc}
\usepackage[spanish]{babel}
\usepackage{amsfonts}
\usepackage{amsthm}

\newtheorem{prop}{Proposición}


\title{Función de orden y algoritmo eficiente de predicción}
\author{Leonardo Andrés Jofré Flor}
\begin{document}
\maketitle

\begin{summary}
El presente trabajo propone una metodología para hacer análisis multivariable
agregando una nueva consideración a la regresión lineal clásica. Esta
consideración es el orden de los valores que se van obteniendo. Esto quiere
decir que si dos variables están relacionadas de forma creciente el orden de las
variables debe ser invariante con respecto la transformación, o por lo menos, no
sebe sufrir cambios significativos, por lo menos en alguna medida. Esta medida
será una nueva definición de covarianza y una nueva definición para medir el
error generado por una estimación de este tipo.
\end{summary}

\begin{prop}
Una función de relación probabilista entre dos funciones $f$ y $g$ es aquella
que considera que si las variables ``estuvieran relacionadas'' es la función que
transforma una de las distribuciones en la otra.
\end{prop}

 Consideremos $f$ función continua en $\left[a,b\right]$, la
función de distribución de las imagenes $f\left(\left[a,b\right]\right)$
está dado por la transformación de variable aleatoria de la
distribución uniforme en $[a,b]$

\begin{prop} La funcion de distrinucion acumulada de una variable
  aleatoria $Y=f\left(X\right)$ donde $X\sim U\left(a,b\right)$  
\end{prop}

$rec(f)\sim F_f$ la función de orden de $f$ está dada por $sort(f) =
f(F_f(f(x)))$

Sea $f:\left[a,b\right]\to\mathbb{R}$

\begin{proof}
	Consideremos $X\sim \mathbb{U}\left(a,b\right)$ con $a<b \in \mathbb{R}$
\end{proof}

\begin{prop}
 La cantidad de relaciones existentes entre las variables es igual a la cantidad
 de mínimos locales existentes que minimizan el error de la combinación lineal
 convexa.
\end{prop}




\end{document} 
