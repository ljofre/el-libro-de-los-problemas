\documentclass{article}

\usepackage[utf8]{inputenc}
\usepackage[spanish]{babel}

\title{Función de orden}
\author{Leonardo Andrés Jofré Flor}
\begin{document}
\maketitle
consideremos $f$ función continua en $[a,b]$, la función de
distribución de las imagenes $f\left([a,b]\right)$ está dado por la
transformación de variable aleatoria de la distribución uniforme en
[a,b]

\begin{def}
  Una funcion ordenada es ... 
\end{def}
Una funcion ordanada comparte algunas propiedades con la funcion
original $f$, por elemplo, la longitud de la funcion $f$ y $\tilda(f)$
debe tener la misma area bajo la curva en el intervalo en donde fueron
ordenadas y la misma longitud de arco.
\begin{theorem} La funcion de distrinucion acumulada de una variable
  aleatoria $Y=f\leftX\rigth)$ donde $X\sim U\left(a,b\rigt)$  
\end{theorem}

$rec(f)\sim F_f$ la función de orden de $f$ está dada por $tilda(f) =
f(F_f(f(x)))$

Sea $f:\[a,b\]\to\mathbb{R}$


\end{document}
