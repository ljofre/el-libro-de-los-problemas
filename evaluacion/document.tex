%%This is a very basic article template.
%%There is just one section and two subsections.
\documentclass{article}
\title{Algoritmo de evaluación}
\begin{document}
\maketitle


\section{Marco teórico}
La capacidad de un alumno no se puede medir en términos absolutos dado una
métrica definida, eso lo podemos apreciar claramente en la situación actual: Un
alumno que obtiene un $7.0$ como calificación no tiene el doble de potencial que
una persona que obtuvo un $3.5$

En un sistema de evaluación correctamente formado se debería cumplir lo
siguiente: Dado que no conocemos de un alumno más que una nota, para un alumno
con mayor capacidad $A_i$ y un alumno con menor capacidad $A_j$ se debe cumplir
la relación
$nota(A_i)>=nota(A_j)$
definición: Capacidad: la capacidad de un alumno no se puede estudiar mediate
una métrica absoluta ya que depende del contexto, eso quiere decir, la
clasificación de ``capaz'' depende netamente de la universidad en donde se esté
estudiando, por lo tanto, se estudia de forma comparativa.
Capacidad es la sistemática repetición de imponer una posición de la evaluación
con respecto sus pares.
\section{Cuantil estable}
El ranking estable de un alumno es la combinación lineal de rankings que
producen que un sistema sea lo más estable posible.
$Q_k = \sum_i p_i b_i r_i$

en donde $\sum p_i = 1$ es una función de ponderación que eliminará los rankings
ruidosos, $b_i$ es una variable booleana que indicará si el alumno cursó la
asignatura y $r_i$ es el ranking obtenido en la asignatura $i$.
\section{Función distancia}
Un profesor encargado de un curso siempre se aleja de una condición ideal, por
lo tanto
$d(R(P_i,\Omega))>=0$
La función distanca debe ser invariante ( o por lo menos robusta) con respecto
la cantidad de alumnos de un curso
 si $A\subsetP_i$ entonces $d(R(P_i,\Omega)) = d(A,\Omega)$

 
 \section{Construcción de la función distancia}
 La función distancia, debe ser
 


\subsection{Subtitle}

Plain text.

\subsection{Another subtitle}

More plain text.


\end{document}
