%%This is a very basic article template.
%%There is just one section and two subsections.

\documentclass{article}

\usepackage{amsfonts}
\usepackage[utf8x]{inputenc}
\usepackage[spanish]{babel}
\usepackage{amsthm}
\usepackage{amsmath}

%% ambientes de teoremas
\newtheorem{problema}{Problema}



\author{Leonardo Andrés Jofré Flor}
\title{Metodo de clasificación probabilista}

\begin{document}
% definir lo que es una tupla
Sea un conjunto aleatorio de tuplas en $T=\{x_i\}$ se busca obtener la función
de densidad de probabilidad $p:\mathbb{R}^n\to[0,1]$

% explicar la relacion entre la función de densidad acumulada y la función de
% orden de los datos
Aproximación de la función de densidad acumulada, caso elemental

% cualquier cambio de base genera una nueva aproximación de clusterización

% la función de probabilidad puede ser representada como combinación lineal

% la función de orden puede aproximar a la distribución acumulada

% solución recursiva

% la función de densidad de puede aproximar a la combinación lineal de funciones

Si consideramos la función indicatriz
$p_\epsilon(x,y)$
\begin{eqnarray*}
1 &si& \|x\|\leq \epsilon\\
0 &si& \|x\| > \epsilon\\
\end{eqnarray*}
$$p_\epsilon(x,y)=1 si \|x\|\le \epsilon$$

$\sum c_{i,j}p_\epsilon(x-x_i,y-y_j)$

% encontrar la base que produce que la función sea solución de una ecuación
% diferencial de Sturm-Liouville ayudaría a encontrar una mejor aproximación

% inversa generalizada de Moore-Penrose

% numero de grupos

% visualización 3D

% Convergencia al número de clusters

\subsection{Anexos}

\begin{problema}
	Sea $$\Phi = \{(x,y)\in \mathbb{R}^2| (x,y) \in
	\texttt{convexhull}\left(p_1,p_2,p_3,p_4 \right) \}$$ encontrar una transformación desde $\Phi$ a $P$
	
\end{problema}

\begin{problema}
	Sea $f:\mathbb{R}^2\to\mathbb{R}$ y sea $\partial A$ la curva orientada en
	sentido antihorario por los punros $p_1,p_2,p_3,p_4$ encontrar el cambio de
	variable adecuado para encontrar
	$$\int_A f(x,y)dA$$
\end{problema}

\begin{problema}
	  
\end{problema}


\end{document}
